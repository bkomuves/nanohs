
\documentclass{beamer}
\mode<presentation>
\usepackage{color}
\usepackage{fancyvrb}
\usefonttheme[onlymath]{serif}

\title{Writing a self-hosting compiler \\ for a purely functional language\\
{\footnotesize (work in progress)}}
\author{Bal\'azs K\H{o}m\H{u}ves \\ {\small [{\tt bkomuves@gmail.com}]}}
\date{2021.03.24.}

%------------------------------------------------------------------------------%

\begin{document}

\begin{frame}
\titlepage
\end{frame}

%------------------------------------------------------------------------------%

\begin{frame}{What is this?}

This is an {\bf experiment} to write a minimalistic self-hosting compiler
in a purely functional language. It's intended both as a learning
experience (for myself) and an experiment in bootstrapping.\\[15pt]

Goals:
\begin{itemize}
\item self-hosting compiler
\item very small code-base
\item bootstrappable using a Haskell implementation
\item generated code should have ``acceptable'' quality \\[15pt]
\end{itemize}

Non-goals:
\begin{itemize}
\item a fully-featured language
\item efficient compilation (the compiler can be slow)
\item bootstrapping ``real'' compilers from this code base
\end{itemize}

\end{frame}

%------------------------------------------------------------------------------%

\begin{frame}{Why self-host?}

Why do bootstrapping / self-hosting?\\[10pt]

\begin{itemize}
\item I like simple, self-contained stuff
\item self-hosting is a philosophically desirable property
\item future-proofness: If you can bootstrap from something simple, it won't bit-rot that easily
\item it's a test of the maturity of the compiler
\item the compiler itself is a very good test program! \\[15pt]
\end{itemize}

Exhibit A: Maru, a tiny self-hosting LISP dialect was successfully revived and
extended by Attila.\\[15pt]

Exhibit B: The York Haskell Compiler (forked from {\tt nhc98}) is now dead, apparently
because \emph{nobody knows how to build it anymore}.

\end{frame}

%------------------------------------------------------------------------------%

\begin{frame}{The language}

The syntax and semantics is based on (strict) Haskell - the idea is that the compiler
should be compilable by both GHC and itself.

This makes development much easier!\\[20pt]

The language is basically untyped call-by-value lambda calculus with data constructors
and recursive lets.
During parsing, Haskell features like type annotations, data type declarations, imports etc
are simply ignored - but the compiler itself is a well-typed program!\\[20pt]

In fact, a type checker should be probably added, after all; but I wanted to keep it simple.

\end{frame}

%------------------------------------------------------------------------------%

\begin{frame}{Language (non-)features}

{\small
\begin{itemize}
\item no static type system (untyped lambda calculus) 
\item na data type declarations (constructors are arbitrary capitalized names)
\item no module system - instead, we have C-style includes
\item strict language (exceptions: if-then-else, logical {\tt and}/{\tt or})
\item ML-style side effects, but wrapped in a monad
\item only simple pattern matching + default branch (TODO: nested patterns)
\item no infix operators
\item list construction syntax {\tt [a,b,c]} is supported
\item no indentation syntax (only curly braces), except for top-level blocks
\item only line comments, starting at the beginning of the line
\item no escaping in character / string constants yet
\item (universal polymorphic equality comparison primop)
\end{itemize}
}

\end{frame}

%------------------------------------------------------------------------------%

\begin{frame}{Compilation pipeline}

\begin{enumerate}
\item lexing and parsing
\item collecting string literals
\item partition recursive lets into strongly connected components
\item TODO: eliminate pattern matching into simple branching on constructors
\item recognize primops
\item collect data constructors
\item scope checking \& conversion to core language
\item TODO: inline small functions + some basic optimizations
\item closure conversion
\item TODO: compile to an SSA intermediate language
\item final code generation 
\end{enumerate}

\end{frame}

%------------------------------------------------------------------------------%

\begin{frame}{The runtime system}

The runtime system is relatively straightforward:

{\small
\begin{itemize}
\item there is a heap, which can only contain closures and data constructors
\item there is stack (separate from the C stack) which contains pointers to the heap
\item heap pointers are tagged using the lowest 3 bits
\item heap objects fitting into 61 bits (ints, chars, nullary constructors, static functions) are
      not allocated, just stored in registers / on stack
\item statically unknown application is implemented as a runtime primitive 
\item there is a simple copying GC - the GC roots are simply the content of the stack
      (plus statically known data)
\item there are some debug features like printing heap objects
\item primops: integer arithmetic; integer comparison; bitwise operations; lazy {\tt ifte/and/or}; basic IO.
\end{itemize}
}

\end{frame}

%------------------------------------------------------------------------------%

\begin{frame}{Current state of the experiment}

Repo: \url{https://github.com/bkomuves/nanohs}\\[15pt]

Current state:
\begin{itemize}
\item it can compile itself successfully;
\item the source code is not as nice as it could be;
\item the generated code is pretty horrible;
\item there are some desirable features missing.\\[15pt]
\end{itemize}

Code size:
\begin{itemize}
\item the compiler: $\sim 1700$ lines (required code)
\item type annotations: $+500$ lines (ignored)
\item the C runtime: $\sim 600$ lines (includes some debug features)
\end{itemize}

\end{frame}

%------------------------------------------------------------------------------%

\begin{frame}{Mistakes I made}

{\small
I made a huge amount of mistakes during the development, which made it much
harder and longer than I expected.

\begin{itemize}
\item Trying to self-host before the compiler actually works. Writing a 
compiler is hard work, you don't want to do it in a limited language!
\item Not figuring out the precise semantics upfront:
   \begin{itemize}
   \item exact details of lazyness vs. strictness
   \item controlling exactly where heap / stack allocations can happen
   \item recursive lets
   \end{itemize}
\item Not having debugging support from early on (for example: good error messages, 
      threading names \& source locations, RTS features, pretty printing AST-s)
\item Trying to target assembly first, instead of a simpler target like C; it's just extra cognitive load
\item Generatic code directly, without having a low-level IL
\item Using de Bruijn indices (as opposed to levels) in generated code
\end{itemize}
}

\end{frame}

%------------------------------------------------------------------------------%

\begin{frame}[fragile]{Example GC bug - Marshalling from C strings}

{\footnotesize
\begin{verbatim}
       heap_obj marshal_from_cstring(char *cstring) { ...
       
v1:    obj    = heap_allocate(Cons,2);
       obj[1] = cstring[0];
       obj[2] = marshal_from_cstring(cstring+1);
       return obj;
       
v2:    rest   = marshal_from_cstring(cstring+1);
       obj    = heap_allocate(Cons,2);
       obj[1] = cstring[0];
       obj[2] = rest;
       return obj;
       
v3:    rest   = marshal_from_cstring(cstring+1);
                push(rest);
       obj    = heap_allocate(Cons,2);
       obj[1] = cstring[0];
       obj[2] = pop();
       return obj;
\end{verbatim}
}

\end{frame}

%------------------------------------------------------------------------------%

\begin{frame}{Future work}

I'm not convinced that it is worth hacking on this toy compiler (as opposed to work
on a full-featured ``real'' compiler), but in any case, here is a TODO list:\\[10pt]

\begin{itemize}
\item add nested patterns, and patterns in lambda arguments
\item do some optimizations, so that the generated code is better
\item tail call elimination
\item escaping in string constants
\item compile to a low-level intermediate language first, and do codegen from that
\item add a type system
\item more backends: x86-64 assembly and/or machine code; some simple VM
\end{itemize}

\end{frame}


%------------------------------------------------------------------------------%

\end{document}
